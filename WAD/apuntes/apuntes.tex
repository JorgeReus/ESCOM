\documentclass{article}
\usepackage{estilo}
\begin{document}
\maketitle
\tableofcontents
\newpage
\section{Introducción}
\subsection{Clase 1-FEB-2018}
\begin{itemize}
	\item Salon del profe: 2108
	\item Correo: \textit{rubenperedo@hotmail.com}
	\item Servidor: \url{http://148.204.53.108/WAD/index.html} de 7:10am-3:00pm
	\item \begin{enumerate}
		\item Introduction to the Web Applications
		\item Servlets
		\item JSP
		\item Frameworks
	\end{enumerate}
	\item ¿Qué es una aplicación WEB?:\\
	Es una aplicación con las siguientes características.
	\begin{itemize}
		\item Usa protocolo http
		\item Usa arquitectura cliente/servidor
		\item Tiene Control Bidireccional
		\item implementa lógica de negocios
	\end{itemize}
	\item Evaluación:
	\begin{itemize}
		\item Examen 1 $\rightarrow$ 25\%
		\item Examen 2 $\rightarrow$ 25\%
		\item CD (15 dìas después del segundo examen) $\rightarrow$ 20\%
		\item Proyecto $\rightarrow$ 30\%
	\end{itemize}
\end{itemize}
\subsection{Clase 2-FEB-2018}
Dependencia de contexto:\\
Que necesita el software para correr\\
La dependecia de contexto más importante de tomcat es \textit{JDK}\\
La configuración está en conf/web.xml
\subsection{Clase 12-FEB-2018}
Servlet : Es una tecnología implementada como una clase de Java, que sirve para generar páginas web dinámicas.\\
Interfaz :  Es un contrato, un esqueleto que se tiene que implementar.\\
No sirve para el comportamiento generico y las patrones de diseño de software.\\
\subsection{Clase 16-FEB-2018}
Los parametros son \textit{request.getParameter(clave);}
Por medio de HttpSessión, podemos guardar objetos en la sesión(Hasta que el navegador se cierre), por medio de session.getAttribute(clave, valor);\\
Por medio de \textit{this.getServletContext(), podemos accesar al nivel de la aplicación, de est manera los objetos que guardemos se quedan en archivos en la aplicación}.\\
Por medio se setearle attributos al request, podemos guardar cosas con scope de request (Ciclo de requestt-Response)
\subsection{Clase 16-FEB-2018}
\textbf{Layer}: Capas lógicas\\
\textbf{Tiers}: Capas físicas\\
Para instalar la aplicación en el servidor Tomcat, se toma el .war y se coloca en la carpeta de tomcat/webapps. Automáticamente lo descomprime
\end{document}
